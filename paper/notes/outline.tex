% Options for packages loaded elsewhere
\PassOptionsToPackage{unicode}{hyperref}
\PassOptionsToPackage{hyphens}{url}
\PassOptionsToPackage{dvipsnames,svgnames,x11names}{xcolor}
%
\documentclass[
  letterpaper,
  DIV=11,
  numbers=noendperiod]{scrartcl}

\usepackage{amsmath,amssymb}
\usepackage{iftex}
\ifPDFTeX
  \usepackage[T1]{fontenc}
  \usepackage[utf8]{inputenc}
  \usepackage{textcomp} % provide euro and other symbols
\else % if luatex or xetex
  \usepackage{unicode-math}
  \defaultfontfeatures{Scale=MatchLowercase}
  \defaultfontfeatures[\rmfamily]{Ligatures=TeX,Scale=1}
\fi
\usepackage{lmodern}
\ifPDFTeX\else  
    % xetex/luatex font selection
\fi
% Use upquote if available, for straight quotes in verbatim environments
\IfFileExists{upquote.sty}{\usepackage{upquote}}{}
\IfFileExists{microtype.sty}{% use microtype if available
  \usepackage[]{microtype}
  \UseMicrotypeSet[protrusion]{basicmath} % disable protrusion for tt fonts
}{}
\makeatletter
\@ifundefined{KOMAClassName}{% if non-KOMA class
  \IfFileExists{parskip.sty}{%
    \usepackage{parskip}
  }{% else
    \setlength{\parindent}{0pt}
    \setlength{\parskip}{6pt plus 2pt minus 1pt}}
}{% if KOMA class
  \KOMAoptions{parskip=half}}
\makeatother
\usepackage{xcolor}
\setlength{\emergencystretch}{3em} % prevent overfull lines
\setcounter{secnumdepth}{-\maxdimen} % remove section numbering
% Make \paragraph and \subparagraph free-standing
\ifx\paragraph\undefined\else
  \let\oldparagraph\paragraph
  \renewcommand{\paragraph}[1]{\oldparagraph{#1}\mbox{}}
\fi
\ifx\subparagraph\undefined\else
  \let\oldsubparagraph\subparagraph
  \renewcommand{\subparagraph}[1]{\oldsubparagraph{#1}\mbox{}}
\fi


\providecommand{\tightlist}{%
  \setlength{\itemsep}{0pt}\setlength{\parskip}{0pt}}\usepackage{longtable,booktabs,array}
\usepackage{calc} % for calculating minipage widths
% Correct order of tables after \paragraph or \subparagraph
\usepackage{etoolbox}
\makeatletter
\patchcmd\longtable{\par}{\if@noskipsec\mbox{}\fi\par}{}{}
\makeatother
% Allow footnotes in longtable head/foot
\IfFileExists{footnotehyper.sty}{\usepackage{footnotehyper}}{\usepackage{footnote}}
\makesavenoteenv{longtable}
\usepackage{graphicx}
\makeatletter
\def\maxwidth{\ifdim\Gin@nat@width>\linewidth\linewidth\else\Gin@nat@width\fi}
\def\maxheight{\ifdim\Gin@nat@height>\textheight\textheight\else\Gin@nat@height\fi}
\makeatother
% Scale images if necessary, so that they will not overflow the page
% margins by default, and it is still possible to overwrite the defaults
% using explicit options in \includegraphics[width, height, ...]{}
\setkeys{Gin}{width=\maxwidth,height=\maxheight,keepaspectratio}
% Set default figure placement to htbp
\makeatletter
\def\fps@figure{htbp}
\makeatother
% definitions for citeproc citations
\NewDocumentCommand\citeproctext{}{}
\NewDocumentCommand\citeproc{mm}{%
  \begingroup\def\citeproctext{#2}\cite{#1}\endgroup}
\makeatletter
 % allow citations to break across lines
 \let\@cite@ofmt\@firstofone
 % avoid brackets around text for \cite:
 \def\@biblabel#1{}
 \def\@cite#1#2{{#1\if@tempswa , #2\fi}}
\makeatother
\newlength{\cslhangindent}
\setlength{\cslhangindent}{1.5em}
\newlength{\csllabelwidth}
\setlength{\csllabelwidth}{3em}
\newenvironment{CSLReferences}[2] % #1 hanging-indent, #2 entry-spacing
 {\begin{list}{}{%
  \setlength{\itemindent}{0pt}
  \setlength{\leftmargin}{0pt}
  \setlength{\parsep}{0pt}
  % turn on hanging indent if param 1 is 1
  \ifodd #1
   \setlength{\leftmargin}{\cslhangindent}
   \setlength{\itemindent}{-1\cslhangindent}
  \fi
  % set entry spacing
  \setlength{\itemsep}{#2\baselineskip}}}
 {\end{list}}
\usepackage{calc}
\newcommand{\CSLBlock}[1]{\hfill\break\parbox[t]{\linewidth}{\strut\ignorespaces#1\strut}}
\newcommand{\CSLLeftMargin}[1]{\parbox[t]{\csllabelwidth}{\strut#1\strut}}
\newcommand{\CSLRightInline}[1]{\parbox[t]{\linewidth - \csllabelwidth}{\strut#1\strut}}
\newcommand{\CSLIndent}[1]{\hspace{\cslhangindent}#1}

\KOMAoption{captions}{tableheading}
\makeatletter
\@ifpackageloaded{caption}{}{\usepackage{caption}}
\AtBeginDocument{%
\ifdefined\contentsname
  \renewcommand*\contentsname{Table of contents}
\else
  \newcommand\contentsname{Table of contents}
\fi
\ifdefined\listfigurename
  \renewcommand*\listfigurename{List of Figures}
\else
  \newcommand\listfigurename{List of Figures}
\fi
\ifdefined\listtablename
  \renewcommand*\listtablename{List of Tables}
\else
  \newcommand\listtablename{List of Tables}
\fi
\ifdefined\figurename
  \renewcommand*\figurename{Figure}
\else
  \newcommand\figurename{Figure}
\fi
\ifdefined\tablename
  \renewcommand*\tablename{Table}
\else
  \newcommand\tablename{Table}
\fi
}
\@ifpackageloaded{float}{}{\usepackage{float}}
\floatstyle{ruled}
\@ifundefined{c@chapter}{\newfloat{codelisting}{h}{lop}}{\newfloat{codelisting}{h}{lop}[chapter]}
\floatname{codelisting}{Listing}
\newcommand*\listoflistings{\listof{codelisting}{List of Listings}}
\makeatother
\makeatletter
\makeatother
\makeatletter
\@ifpackageloaded{caption}{}{\usepackage{caption}}
\@ifpackageloaded{subcaption}{}{\usepackage{subcaption}}
\makeatother
\ifLuaTeX
  \usepackage{selnolig}  % disable illegal ligatures
\fi
\usepackage{bookmark}

\IfFileExists{xurl.sty}{\usepackage{xurl}}{} % add URL line breaks if available
\urlstyle{same} % disable monospaced font for URLs
\hypersetup{
  pdftitle={Paper Outline},
  colorlinks=true,
  linkcolor={blue},
  filecolor={Maroon},
  citecolor={Blue},
  urlcolor={Blue},
  pdfcreator={LaTeX via pandoc}}

\title{Paper Outline}
\author{}
\date{}

\begin{document}
\maketitle

\section{Introduction}\label{introduction}

\begin{enumerate}
\def\labelenumi{\arabic{enumi}.}
\item
  Characterization of Inland Water Bodies via Remote Sensing is
  Challenging

  \begin{itemize}
  \tightlist
  \item
    complex spectral signatures
  \item
    limited spatial and temporal resolution (long satellite overpasses)
  \item
    nevertheless remote sensing has significant potential
  \item
    Harmful Algal Bloom (HAB) classification and mapping
  \item
    Oil Spill mapping
  \end{itemize}
\item
  Hyperspectral Imaging addresses spectral limitations of traditional
  remote sensing platforms
\item
  Supervised methods

  \begin{itemize}
  \tightlist
  \item
    Inversion of optically-active parameters
  \item
    Key limitation: availability of in-situ reference data

    \begin{itemize}
    \tightlist
    \item
      Cite Aurin et al.~and Ross et al.~here
    \end{itemize}
  \item
    When ground truth data are limited, classification is an alternative
    approach, e.g.~``bad'' to ``good'' water quality
  \end{itemize}
\item
  Unsupervised methods

  \begin{itemize}
  \tightlist
  \item
    Unsupervised classification \& ML Techniques
  \item
    Endmember Extraction and Spectral Unmixing
  \end{itemize}
\item
  Robot Teams and HSI equipped Drones
\item
  This paper

  \begin{itemize}
  \tightlist
  \item
    Extend capabilities of robot team by showcasing capabilities for
    unsupervised classification
  \item
    Utilize GTM as Bayesian extension of SOM to enable unsupervised
    classification
  \item
    GTM also enables extraction of spectral features associated with
    each node
  \item
    Two case studies

    \begin{enumerate}
    \def\labelenumii{\arabic{enumii}.}
    \tightlist
    \item
      GTM trained on water-only HSI data for lake

      \begin{itemize}
      \tightlist
      \item
        Identify interesting regions for further examination by robot
        team
      \item
        Investigate spectral variability of water
      \end{itemize}
    \item
      GTM on dataset combining grass and water pixels with a simulated
      pollution event using Rhodamine tracer dye

      \begin{itemize}
      \tightlist
      \item
        Use extracted GTM ``endmembers'' together with NS3 to map
        abundance
      \item
        Identify abundance of algae
      \item
        Map evolution of Rhodamine dye plume
      \end{itemize}
    \end{enumerate}
  \end{itemize}
\end{enumerate}

\section{Materials and Methods}\label{materials-and-methods}

\begin{enumerate}
\def\labelenumi{\arabic{enumi}.}
\tightlist
\item
  Data Collection

  \begin{itemize}
  \tightlist
  \item
    Describe collection at site in Montague, North Texas
  \end{itemize}
\item
  Autonomous Robot Team

  \begin{itemize}
  \tightlist
  \item
    Describe drone

    \begin{itemize}
    \tightlist
    \item
      Alta X Quadcopter
    \item
      Resonon Pika XC2 HSI
    \item
      Intel Nuc compute
    \end{itemize}
  \item
    describe georectification procedure
  \end{itemize}
\item
  Pre-Processing
\item
  Generative Topographic Mapping
\end{enumerate}

\begin{itemize}
\tightlist
\item
  Original SOM paper (Kohonen 1990)
\item
  Heskes reinterpretation of SOM algorithm as stochastic gradient
  descent on an energy suface i.e.~justification with a cost function
  (Heskes 1999)
\item
  GTM paper (Bishop, Svensén, and Williams 1998b)

  \begin{itemize}
  \tightlist
  \item
    Describe algorithm
  \item
    Describe parameters
  \end{itemize}
\item
  Bayesian Information Criterion and Akaike Information Criterion for
  Hyperparameter Optimization

  \begin{itemize}
  \tightlist
  \item
    A full Bayesian treatment would involve marginalization over all
    possible parameter values but this is much more computational
    efficient.
  \end{itemize}
\end{itemize}

\begin{enumerate}
\def\labelenumi{\arabic{enumi}.}
\setcounter{enumi}{4}
\tightlist
\item
  Abundance Mapping with the Normalized Spectral Similarity Score

  \begin{itemize}
  \tightlist
  \item
    Describe different metrics for comparison of spectra

    \begin{itemize}
    \tightlist
    \item
      Trade-off between metrics that distinguish changes in overall
      intensity and changes in peak location, i.e.~hue
    \end{itemize}
  \end{itemize}
\end{enumerate}

\section{Results}\label{results}

\begin{enumerate}
\def\labelenumi{\arabic{enumi}.}
\tightlist
\item
  Two Case Studies
\item
  GTM Fit on Full Dataset
\item
  Hyperparameter Optimization with BIC
\item
  Spectral Signature Identification
\item
  Water Class Map (Water Only GTM)
\item
  Algae Identification
\item
  Dye Plume Identification
\end{enumerate}

\section{Discussion}\label{discussion}

\begin{enumerate}
\def\labelenumi{\arabic{enumi}.}
\tightlist
\item
  GTM Applications in our Paper
\item
  Comparison to existing approaches

  \begin{itemize}
  \tightlist
  \item
    Unsupervised Classification
  \item
    Endmember Extraction \& Spectral Unmixing
  \item
    Applications for either of these with Drones?
  \end{itemize}
\item
  Use of GTM and Unsupervised methods for Robot Team

  \begin{itemize}
  \tightlist
  \item
    As a feature transformer/preprocessor for supervised methods (as
    opposed to, say, PCA or k-NN)
  \item
    identification of ``interesting'' regions for intelligent deployment
    of robotic boat

    \begin{itemize}
    \tightlist
    \item
      don't continue collecting data in region that is similar
    \item
      provision boat to maximize data collection across distribution of
      GTM classes
    \end{itemize}
  \end{itemize}
\item
  Extensions and future work

  \begin{itemize}
  \tightlist
  \item
    Incorporation of GTM extensions for efficient online-learning

    \begin{itemize}
    \tightlist
    \item
      Currently GTM classes can quickly be applied (in real time) once
      they are trained to generate abundance maps e.g.~chlorophyll,
      algae, etc\ldots{}
    \item
      Key limitation of GTM is similar to SOM and k-NN: you need to
      compute distances between samples and latent nodes. This can scale
      poorly for very large datasets
    \item
      Bishop suggests a way to augment GTM training algorithm to enable
      online training of GTM. This could be deployed on-board the UAV to
      perform classification and identify spectral signatures in
      near-real time.
    \item
      Extensions of GTM (Bishop, Svensén, and Williams 1998a)
    \end{itemize}
  \item
    Utilize the GTM with in-situ data collection to identify spectral
    signatures of specific algal species
  \end{itemize}
\end{enumerate}

\section{Conclusions}\label{conclusions}

\section{NOTES AND CITATIONS}\label{notes-and-citations}

\begin{enumerate}
\def\labelenumi{\arabic{enumi}.}
\tightlist
\item
  Applications of remote sensing to water quality

  \begin{itemize}
  \tightlist
  \item
    machine learning (XGBoost) used with paired remote sensing imagery
    and in-situ data to classify imagery into 5 categories including 3
    for harmful algal blooms (Ghatkar, Singh, and Shanmugam 2019)
  \item
    Spectral signatures of classes are complex and often overlapping
    (Thenkabail, Lyon, and Huete 2018)
  \item
    Remote sensing used for oil spill analysis: extent and thickness
    mapping (Kokaly et al. 2013; Leifer et al. 2012)
  \item
    Sun glitter remains a key challenge for sensing in the visible
    portion of the spectrum but multi and hyperspectral imagers have
    been used for oil spill identification and to identify their impacts
    on vegetation stress and mortality (Fingas and Brown 2014; Khan et
    al. 2018)
  \end{itemize}
\item
  Hyperspectral Imagery

  \begin{itemize}
  \tightlist
  \item
    applications include food quality \& safety, medical diagnoses,
    precision agriculture, and forensic document examination (Khan et
    al. 2018)
  \item
    Hyperspectral data were used to distinguish oils by type,
    e.g.~crude, diesel, gasoline, and palm (Yang et al. 2020)
  \end{itemize}
\item
  Supervised regression and classification for water quality

  \begin{itemize}
  \tightlist
  \item
    common approach is inversion of optically-active water quality
    parameters such as chlorophyll-a, blue-green algae, turbidity, and
    temperature (Ritchie, Zimba, and Everitt 2003)
  \item
    Combining spectral indices such as the NDVI together with machine
    learning is a popular approach (Thenkabail, Lyon, and Huete 2018;
    Sagan et al. 2020; Lu et al. 2021)
  \item
    Polynomial regression models for chlorophyll-a, turbidity (D. Zhang,
    Zeng, and He 2022)
  \item
    Key limitation is collection of sufficient quantity of in-situ
    reference data
  \item
    Ross et al.~created a comprehensive dataset with over 600,000 water
    quality records matching optically active water quality parameters
    with associated satellite imagery from Landsat 5,7,8. To achieve
    this quantity of data, they needed records spanning 1984 to 2019
    (Ross et al. 2019)
  \item
    Aurin et al.~took a similar route combining 30 years of remote
    sensing imagery with in-situ data from over 500 field compaigns for
    CDOM, organic carbon, etc\ldots{} (Aurin, Mannino, and Lary 2018).
  \item
    When ground truth data aren't available in sufficient quantity,
    classification into water quality categories is another approach
    (ground truth can be easier to obtain by expert analysis of scene)
    (Koponen et al. 2002)
  \item
    When no ground-truth data are available, unsupervised classification
    can still help partition imagery into groups or clusters.
  \item
    Many data-driven, ML methods have been employed for the task
    including various matrix factorizations, k-nearest neighbors, fuzzy
    c-means, density estimation methods, etc. (L. Zhang et al. 2019)
  \item
    Additionally unsupervised approaches can be used to perform
    nonlinear dimensionality reduction \& pre-processing for supervised
    approaches.
  \item
    SOM used for remote sensing imagery classification (Wan and Fraser
    2000)
  \item
    SOM used for clustering and data compression of HSI cube-sat
    (Danielsen, Johansen, and Garrett 2021)
  \item
    SOM used for land-use and land-cover change analysis (Penfound and
    Vaz 2021)
  \end{itemize}
\item
  Endmember Extraction and Spectral Unmixing

  \begin{itemize}
  \tightlist
  \item
    This is a related problem where the goal is to identify unique
    spectral signatures which combine (linearly or non-linearly) to
    produce the measured signal.
  \item
    Having identified the set of endmembers, we then seek to determine
    their relative abundance in each pixel
  \item
    Many statistical approaches for extracting endmembers (Berman et al.
    2004)
  \item
    Sparse PCA used for endmember extraction (Yousefi et al. 2016)
  \item
    Popular spectral unmixing approach is to treat ``pure'' endmembers
    as vertices of a simplex {[}Plaza et al. (2012);
    nascimento2005vertex{]}
  \item
    Convolutional Neural Networks with Autoencoder architectures are a
    popular ML approach which identify endmembers and perform non-linear
    unmixing (Palsson, Ulfarsson, and Sveinsson 2020; Su et al. 2017,
    2019; Borsoi, Imbiriba, and Bermudez 2019)
  \item
    Self Organizing Map is another approach which has been used for
    endmember extraction together with neural networks for abundance
    mapping (unmixing) (Cantero et al. 2004)
  \item
    SOM has bee nused for identifying synoptic-scale patterns in wind
    and sea surface temperature data (Richardson, Risien, and
    Shillington 2003)
  \item
    SOM used for HSI feature extraction (Ceylan and Kaya 2021)
  \item
    drawback of autoencoder and other statistical approach is lack of a
    (topological) relationship between the classes/endmembers. Is
    endmember 1 closer to endmember 2 or endmember 10? The SOM addresses
    this. The GTM is even better\ldots{}
  \end{itemize}
\item
  Abundance Mapping

  \begin{itemize}
  \tightlist
  \item
    Many different spectral similarity functions exist for comparing
    spectra. Each have trade-offs between ability to distinguish
    differences in intensity versus differences in peak location (hue)
    (Deborah, Richard, and Hardeberg 2015)
  \item
    spectral angle mapper is popular similarity function used in
    endmember extraction. The SAM is nice because it is more sensitive
    to shape than scale (Jiang, Werff, and Meer 2020)
  \item
    spectral correlation mapper introduced to as a statistical
    alternative to SAM based on covariance instead of spectral angle (De
    Carvalho and Meneses 2000)
  \item
    Normalized spectral similarity score developed by Nidamanuri et
    al.~combine the mean-squared error with spectral angle for a happy
    medium (Nidamanuri and Zbell 2011)
  \end{itemize}
\item
  Drone-based HSI

  \begin{itemize}
  \tightlist
  \item
    Near-earth HSI addresses spatial, spectral, and temporal limitations
    of satellite and airborne platforms. UAV-based HSI enable fine-scale
    mapping (Banerjee, Raval, and Cullen 2020)
  \item
    Drones can be equipped with HSI and compute to enable rapid
    generation of spectral indices like the NDVI for applications such
    as precision agriculture (Horstrand et al. 2019)
  \item
    UAV-based HSI can be georectified to centimeter-scales without need
    for ground control points by using on-board GPS and IMU (Arroyo-Mora
    et al. 2019)
  \item
    UAV-based HSI for turbidity estimation (Vogt and Vogt 2016)
  \end{itemize}
\end{enumerate}

\phantomsection\label{refs}
\begin{CSLReferences}{1}{0}
\bibitem[\citeproctext]{ref-arroyo2019implementation}
Arroyo-Mora, J Pablo, Margaret Kalacska, Deep Inamdar, Raymond Soffer,
Oliver Lucanus, Janine Gorman, Tomas Naprstek, et al. 2019.
{``Implementation of a UAV--Hyperspectral Pushbroom Imager for
Ecological Monitoring.''} \emph{Drones} 3 (1): 12.

\bibitem[\citeproctext]{ref-aurin2018remote}
Aurin, Dirk, Antonio Mannino, and David J Lary. 2018. {``Remote Sensing
of CDOM, CDOM Spectral Slope, and Dissolved Organic Carbon in the Global
Ocean.''} \emph{Applied Sciences} 8 (12): 2687.

\bibitem[\citeproctext]{ref-banerjee2020uav}
Banerjee, Bikram Pratap, Simit Raval, and PJ Cullen. 2020.
{``UAV-Hyperspectral Imaging of Spectrally Complex Environments.''}
\emph{International Journal of Remote Sensing} 41 (11): 4136--59.

\bibitem[\citeproctext]{ref-berman2004ice}
Berman, Mark, Harri Kiiveri, Ryan Lagerstrom, Andreas Ernst, Rob Dunne,
and Jonathan F Huntington. 2004. {``ICE: A Statistical Approach to
Identifying Endmembers in Hyperspectral Images.''} \emph{IEEE
Transactions on Geoscience and Remote Sensing} 42 (10): 2085--95.

\bibitem[\citeproctext]{ref-gtm-bishop-2}
Bishop, Christopher M, Markus Svensén, and Christopher KI Williams.
1998a. {``Developments of the Generative Topographic Mapping.''}
\emph{Neurocomputing} 21 (1-3): 203--24.

\bibitem[\citeproctext]{ref-gtm-bishop-1}
---------. 1998b. {``GTM: The Generative Topographic Mapping.''}
\emph{Neural Computation} 10 (1): 215--34.

\bibitem[\citeproctext]{ref-borsoi2019deep}
Borsoi, Ricardo Augusto, Tales Imbiriba, and José Carlos Moreira
Bermudez. 2019. {``Deep Generative Endmember Modeling: An Application to
Unsupervised Spectral Unmixing.''} \emph{IEEE Transactions on
Computational Imaging} 6: 374--84.

\bibitem[\citeproctext]{ref-cantero2004analysis}
Cantero, MC, RM Perez, Pablo J Martinez, PL Aguilar, Javier Plaza, and
Antonio Plaza. 2004. {``Analysis of the Behavior of a Neural Network
Model in the Identification and Quantification of Hyperspectral
Signatures Applied to the Determination of Water Quality.''} In
\emph{Chemical and Biological Standoff Detection II}, 5584:174--85.
SPIE.

\bibitem[\citeproctext]{ref-som-hsi}
Ceylan, Oğuzhan, and Gülsen Taskin Kaya. 2021. {``Feature Selection
Using Self Organizing Map Oriented Evolutionary Approach.''} \emph{2021
IEEE International Geoscience and Remote Sensing Symposium IGARSS},
4003--6. \url{https://api.semanticscholar.org/CorpusID:238750026}.

\bibitem[\citeproctext]{ref-danielsen2021self}
Danielsen, Aksel S, Tor Arne Johansen, and Joseph L Garrett. 2021.
{``Self-Organizing Maps for Clustering Hyperspectral Images on-Board a
Cubesat.''} \emph{Remote Sensing} 13 (20): 4174.

\bibitem[\citeproctext]{ref-de2000spectral}
De Carvalho, O Abilio, and Paulo Roberto Meneses. 2000. {``Spectral
Correlation Mapper (SCM): An Improvement on the Spectral Angle Mapper
(SAM).''} In \emph{Summaries of the 9th JPL Airborne Earth Science
Workshop, JPL Publication 00-18}, 9:2. JPL publication Pasadena, CA,
USA.

\bibitem[\citeproctext]{ref-deborah2015comprehensive}
Deborah, Hilda, Noël Richard, and Jon Yngve Hardeberg. 2015. {``A
Comprehensive Evaluation of Spectral Distance Functions and Metrics for
Hyperspectral Image Processing.''} \emph{IEEE Journal of Selected Topics
in Applied Earth Observations and Remote Sensing} 8 (6): 3224--34.

\bibitem[\citeproctext]{ref-fingas2014review}
Fingas, Merv, and Carl Brown. 2014. {``Review of Oil Spill Remote
Sensing.''} \emph{Marine Pollution Bulletin} 83 (1): 9--23.

\bibitem[\citeproctext]{ref-ghatkar2019classification}
Ghatkar, Jayesh Ganpat, Rakesh Kumar Singh, and Palanisamy Shanmugam.
2019. {``Classification of Algal Bloom Species from Remote Sensing Data
Using an Extreme Gradient Boosted Decision Tree Model.''}
\emph{International Journal of Remote Sensing} 40 (24): 9412--38.

\bibitem[\citeproctext]{ref-som-cost-function}
Heskes, Tom. 1999. {``Energy Functions for Self-Organizing Maps.''} In
\emph{Kohonen Maps}, 303--15. Elsevier.

\bibitem[\citeproctext]{ref-horstrand2019uav}
Horstrand, Pablo, Raúl Guerra, Aythami Rodrı́guez, Marı́a Dı́az, Sebastián
López, and José Fco López. 2019. {``A UAV Platform Based on a
Hyperspectral Sensor for Image Capturing and on-Board Processing.''}
\emph{IEEE Access} 7: 66919--38.

\bibitem[\citeproctext]{ref-jiang2020classification}
Jiang, Tingxuan, Harald van der Werff, and Freek van der Meer. 2020.
{``Classification Endmember Selection with Multi-Temporal Hyperspectral
Data.''} \emph{Remote Sensing} 12 (10): 1575.

\bibitem[\citeproctext]{ref-khan2018modern}
Khan, Muhammad Jaleed, Hamid Saeed Khan, Adeel Yousaf, Khurram Khurshid,
and Asad Abbas. 2018. {``Modern Trends in Hyperspectral Image Analysis:
A Review.''} \emph{Ieee Access} 6: 14118--29.

\bibitem[\citeproctext]{ref-kohonen-som-1}
Kohonen, Teuvo. 1990. {``The Self-Organizing Map.''} \emph{Proceedings
of the IEEE} 78 (9): 1464--80.

\bibitem[\citeproctext]{ref-kokaly2013spectroscopic}
Kokaly, Raymond F, Brady R Couvillion, JoAnn M Holloway, Dar A Roberts,
Susan L Ustin, Seth H Peterson, Shruti Khanna, and Sarai C Piazza. 2013.
{``Spectroscopic Remote Sensing of the Distribution and Persistence of
Oil from the Deepwater Horizon Spill in Barataria Bay Marshes.''}
\emph{Remote Sensing of Environment} 129: 210--30.

\bibitem[\citeproctext]{ref-koponen2002lake}
Koponen, Sampsa, Jouni Pulliainen, Kari Kallio, and Martti Hallikainen.
2002. {``Lake Water Quality Classification with Airborne Hyperspectral
Spectrometer and Simulated MERIS Data.''} \emph{Remote Sensing of
Environment} 79 (1): 51--59.

\bibitem[\citeproctext]{ref-leifer2012state}
Leifer, Ira, William J Lehr, Debra Simecek-Beatty, Eliza Bradley, Roger
Clark, Philip Dennison, Yongxiang Hu, et al. 2012. {``State of the Art
Satellite and Airborne Marine Oil Spill Remote Sensing: Application to
the BP Deepwater Horizon Oil Spill.''} \emph{Remote Sensing of
Environment} 124: 185--209.

\bibitem[\citeproctext]{ref-lu2021retrieval}
Lu, Qikai, Wei Si, Lifei Wei, Zhongqiang Li, Zhihong Xia, Song Ye, and
Yu Xia. 2021. {``Retrieval of Water Quality from UAV-Borne Hyperspectral
Imagery: A Comparative Study of Machine Learning Algorithms.''}
\emph{Remote Sensing} 13 (19): 3928.

\bibitem[\citeproctext]{ref-nidamanuri2010normalized}
Nidamanuri, Rama Rao, and Bernd Zbell. 2011. {``Normalized Spectral
Similarity Score ( \({\hbox{NS}}^{3}\)) as an Efficient Spectral Library
Searching Method for Hyperspectral Image Classification.''} \emph{IEEE
Journal of Selected Topics in Applied Earth Observations and Remote
Sensing} 4: 226--40.
\url{https://api.semanticscholar.org/CorpusID:32567143}.

\bibitem[\citeproctext]{ref-palsson2020convolutional}
Palsson, Burkni, Magnus O Ulfarsson, and Johannes R Sveinsson. 2020.
{``Convolutional Autoencoder for Spectral--Spatial Hyperspectral
Unmixing.''} \emph{IEEE Transactions on Geoscience and Remote Sensing}
59 (1): 535--49.

\bibitem[\citeproctext]{ref-penfound2021analysis}
Penfound, Elissa, and Eric Vaz. 2021. {``Analysis of Wetland Landcover
Change in Great Lakes Urban Areas Using Self-Organizing Maps.''}
\emph{Remote Sensing} 13 (24): 4960.

\bibitem[\citeproctext]{ref-plaza2012endmember}
Plaza, Javier, Eligius MT Hendrix, Inmaculada Garcı́a, Gabriel Martı́n,
and Antonio Plaza. 2012. {``On Endmember Identification in Hyperspectral
Images Without Pure Pixels: A Comparison of Algorithms.''} \emph{Journal
of Mathematical Imaging and Vision} 42: 163--75.

\bibitem[\citeproctext]{ref-som-satellite}
Richardson, Anthony J, C Risien, and Frank Alan Shillington. 2003.
{``Using Self-Organizing Maps to Identify Patterns in Satellite
Imagery.''} \emph{Progress in Oceanography} 59 (2-3): 223--39.

\bibitem[\citeproctext]{ref-ritchie2003remote}
Ritchie, Jerry C, Paul V Zimba, and James H Everitt. 2003. {``Remote
Sensing Techniques to Assess Water Quality.''} \emph{Photogrammetric
Engineering \& Remote Sensing} 69 (6): 695--704.

\bibitem[\citeproctext]{ref-ross2019aquasat}
Ross, Matthew RV, Simon N Topp, Alison P Appling, Xiao Yang, Catherine
Kuhn, David Butman, Marc Simard, and Tamlin M Pavelsky. 2019.
{``AquaSat: A Data Set to Enable Remote Sensing of Water Quality for
Inland Waters.''} \emph{Water Resources Research} 55 (11): 10012--25.

\bibitem[\citeproctext]{ref-sagan2020monitoring}
Sagan, Vasit, Kyle T Peterson, Maitiniyazi Maimaitijiang, Paheding
Sidike, John Sloan, Benjamin A Greeling, Samar Maalouf, and Craig Adams.
2020. {``Monitoring Inland Water Quality Using Remote Sensing: Potential
and Limitations of Spectral Indices, Bio-Optical Simulations, Machine
Learning, and Cloud Computing.''} \emph{Earth-Science Reviews} 205:
103187.

\bibitem[\citeproctext]{ref-su2019daen}
Su, Yuanchao, Jun Li, Antonio Plaza, Andrea Marinoni, Paolo Gamba, and
Somdatta Chakravortty. 2019. {``DAEN: Deep Autoencoder Networks for
Hyperspectral Unmixing.''} \emph{IEEE Transactions on Geoscience and
Remote Sensing} 57 (7): 4309--21.

\bibitem[\citeproctext]{ref-non-negative-autoencoders}
Su, Yuanchao, Andrea Marinoni, Jun Li, Antonio Plaza, and Paolo Gamba.
2017. {``Nonnegative Sparse Autoencoder for Robust Endmember Extraction
from Remotely Sensed Hyperspectral Images.''} In \emph{2017 IEEE
International Geoscience and Remote Sensing Symposium (IGARSS)}, 205--8.
\url{https://doi.org/10.1109/IGARSS.2017.8126930}.

\bibitem[\citeproctext]{ref-thenkabail2018hyperspectral}
Thenkabail, Prasad S, John G Lyon, and Alfredo Huete. 2018.
\emph{Hyperspectral Indices and Image Classifications for Agriculture
and Vegetation}. CRC press.

\bibitem[\citeproctext]{ref-vogt2016near}
Vogt, Michael C, and Mark E Vogt. 2016. {``Near-Remote Sensing of Water
Turbidity Using Small Unmanned Aircraft Systems.''} \emph{Environmental
Practice} 18 (1): 18--31.

\bibitem[\citeproctext]{ref-msom-remote-sensing}
Wan, Weijian, and Donald Fraser. 2000. {``A Multiple Self-Organizing Map
Scheme for Remote Sensing Classification.''} In \emph{International
Workshop on Multiple Classifier Systems}, 300--309. Springer.

\bibitem[\citeproctext]{ref-yang2020characterization}
Yang, Junfang, Jianhua Wan, Yi Ma, Jie Zhang, and Yabin Hu. 2020.
{``Characterization Analysis and Identification of Common Marine Oil
Spill Types Using Hyperspectral Remote Sensing.''} \emph{International
Journal of Remote Sensing} 41 (18): 7163--85.

\bibitem[\citeproctext]{ref-yousefi2016mineral}
Yousefi, Bardia, Saeed Sojasi, Clemente Ibarra Castanedo, Georges
Beaudoin, François Huot, Xavier PV Maldague, Martin Chamberland, and
Erik Lalonde. 2016. {``Mineral Identification in Hyperspectral Imaging
Using Sparse-PCA.''} In \emph{Thermosense: Thermal Infrared Applications
XXXVIII}, 9861:312--22. SPIE.

\bibitem[\citeproctext]{ref-zhang2022selection}
Zhang, Dingyu, Siyu Zeng, and Weiqi He. 2022. {``Selection and
Quantification of Best Water Quality Indicators Using UAV-Mounted
Hyperspectral Data: A Case Focusing on a Local River Network in Suzhou
City, China.''} \emph{Sustainability} 14 (23): 16226.

\bibitem[\citeproctext]{ref-zhang2019hyperspectral}
Zhang, Lefei, Liangpei Zhang, Bo Du, Jane You, and Dacheng Tao. 2019.
{``Hyperspectral Image Unsupervised Classification by Robust Manifold
Matrix Factorization.''} \emph{Information Sciences} 485: 154--69.

\end{CSLReferences}



\end{document}
