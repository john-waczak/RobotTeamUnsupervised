\documentclass{article}
\usepackage{graphicx} % Required for inserting images
\usepackage[margin=0.75in]{geometry}
\usepackage{outlines}


\begin{document}

\section{Introduction}

Inland water bodies present a unique challenge to characterization by remote sensing imagery due to their complex spectral characteristics and small scale spatial variability. The broad bands of multi-spectral imagers coupled with the irregular shape of lakes and rivers results in pixels with highly mixed signals that are easily dominated by reflectance from shore and near-shore vegetation sources \cite{koponen2002lake, ritchie2003remote}. Recently, the combination of hyperspectral imaging with unmanned aerial vehicles (UAV) such as drones has emerged as a powerful approach to simultaneously address the spectral, spatial, and temporal limitations of traditional high-altitude and satellite-based collection \cite{adao2017hyperspectral,arroyo2019implementation}. UAV are significantly less expensive to deploy than their remote sensing counterparts, and low altitude flights enable centimeter-scale sampling while limiting the need for complicated atmospheric corrections \cite{banerjee2020uav}. However, the significant increase in data volume generated by these systems presents a new challenge, namely, how to efficiently extract water quality parameters of interest from intricate pixel spectra.

Extensive research efforts have been dedicated to the development of techniques and algorithms to retrieve water quality parameters from UAV-captured hyperspectral images (HSI). On-board compute installed alongside hyperspectral imagers can enable the rapid evaluation of spectral indices from HSI band ratios \cite{horstrand2019uav}. These band ratios and polynomial combinations of bands have been used to successfully invert optically-active water quality parameters such as turbidity directly from UAV acquired imagery \cite{vogt2016near, zhang2022selection}. Supervised machine learning techniques such as tree-based models, support vector machines, and neural networks have also been used to estimate a plethora of parameters such as colored dissolved organic matter, chlorophyll a, blue-green algae, and suspended sediment concentrations \cite{keller2018hyperspectral, lu2021retrieval}. The calibration and evaluation of these data-driven models demand a significant volume of coincident, in situ data. This can be addressed by coordinating UAV flights with reference data collection using autonomous robotic boats \cite{robot-team-1, robot-team-2}. Nevertheless, this approach relies on prior knowledge of expected sources in order to select appropriate reference instruments for model validation. The presence of unanticipated contaminants cannot be directly identified in this paradigm. 

% Unsupervised methods

To address this limitation, unsupervised methods that segment HSI pixels and can identify unique spectral signatures are needed. 

These methods generally fall into two categories: dimensionality reduction and endmember extraction. For example, principle component analysis (PCA) and t-distributed stochastic neighbor embedding (tSNE) make it possible to reduce high dimensional HSI into two or three dimensions where they can be visualized and compared \cite{}. 

In remote sensing, where reference data is sparse, many approaches have been explored and generally fall into two categories: dimensionality reduction, and unsupervised classification and endmember extraction. 

Algorithms such as principal component analysis (PCA) and t-distributed stochastic neighbor embedding (t-SNE) enable the visualization of HSI by reducing their dimensionality \cite{yousefi2016mineral}.

including dimensionality reduction techniques like principal component analysis (PCA) and clustering algorithms such as k-nearest neighbors \cite{yousefi2016mineral,cariou2015unsupervised}. Non-negative matrix factorization is another popular approach which seeks to simultaneously extract endmembers with their associated abundances via a constrained matrix decomposition \cite{Feng2022HyperspectralUB}. Similarly, the advent of deep learning has led to a profusion of neural network based models utilizing an autoencoder architecture to enable nonlinear endmember extraction \cite{su2019daen,borsoi2019deep,palsson2020convolutional}. Other methods such as vertex component analysis (VCA) are tailored specifically for hyperspectral data based on the observation that endmember spectra form the vertices of a simplex \cite{nascimento2005vertex}. Ultimately, an ideal approach for near-real time analysis of water quality from HSI should enable both visualization via dimensionality reduction and identification of relevant spectral endmembers.

% 4. SOM 

- SOM is an attractive approach as it enables both dimensionality reduction and unsupervised classification. 
- Furthermore, the SOM maintains a topological relationship between learned classes such that classes near each other in the SOM grid are more similar than classes far apart
- SOM can be used to nonlinearly map data into a 2-dimensional space
- Weight vectors corresponding to each SOM node can be interpreted as spectral endmembers. The training algorithm updates weights based on euclidean distance between data record and each node weight vector
- Add citations and examples of use in remote sensing / HSI
- 

Among unsupervised methods, Kohonen's self-organizing map provides an attractive compromise enabling both dimensionality reduction and classification...

% 5. GTM

% 6. This paper...


Using data collected at site in Montague, North Texas, we demonstrate the ability of this paradigm with two case studies. First, we demonstrate how the GTM can be used to provide a high-resolution unsupervised classification of a water body enabling the rapid identification of relevant regions which can then be used to direct further investigation by reference-sensor equipped boats. Second, we demonstrate the ability to perform endmember extraction and abundance mapping using imagery from a test-release of rhodamine dye. Together with the normalized spectral similarity score, these spectral signatures are identified and their associated area in the scene is determined.




\bibliographystyle{plain}
\bibliography{paper/references.bib}

\end{document}