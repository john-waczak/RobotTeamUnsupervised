\documentclass{article}
\usepackage{graphicx} % Required for inserting images
\usepackage[margin=0.75in]{geometry}
\usepackage{outlines}


\begin{document}

\section{Introduction}

Inland water bodies present a unique challenge to characterization by remote sensing imagery due to their complex spectral characteristics and small scale spatial variability. The broad bands of multi-spectral imagers coupled with the irregular shape of lakes and rivers results in pixels with highly mixed signals that are easily dominated by reflectance from shore and near-shore vegetation sources \cite{koponen2002lake, ritchie2003remote}. Recently, the combination of hyperspectral imaging with unmanned aerial vehicles (UAV) such as drones has emerged as a powerful approach to simultaneously address the spectral, spatial, and temporal limitations of traditional high-altitude and satellite-based collection \cite{adao2017hyperspectral,arroyo2019implementation}. UAV are significantly less expensive to deploy than their remote sensing counterparts, and low altitude flights enable centimeter-scale sampling while limiting the need for complicated atmospheric corrections \cite{banerjee2020uav}. However, the significant increase in data volume generated by these systems presents a new challenge, namely, how to efficiently extract water quality parameters of interest from intricate pixel spectra.

Significant research efforts have focused on the development of techniques and algorithms to retrieve water quality parameters from UAV-captured hyperspectral images (HSI). On-board compute installed alongside hyperspectral imagers can enable the rapid evaluation of spectral indices from HSI band ratios \cite{horstrand2019uav}. These band ratios and polynomial combinations of bands have been used to successfully invert optically-active water quality parameters such as turbidity directly from UAV acquired imagery \cite{vogt2016near, zhang2022selection}. Supervised machine learning techniques such as tree-based models, support vector machines, and neural networks have also been used to estimate a plethora of parameters such as colored dissolved organic matter, chlorophyll a, blue-green algae, and suspended sediment concentrations \cite{keller2018hyperspectral, lu2021retrieval}. The calibration and evaluation of these data-driven models demand a significant volume of coincident, in situ data. This can be addressed by coordinating UAV flights with reference data collection using autonomous robotic boats \cite{robot-team-1, robot-team-2}. Nevertheless, this approach relies on prior knowledge of expected sources in order to select appropriate reference instruments for model validation. The presence of unanticipated contaminants cannot be directly identified in this sensing paradigm. 

Extending the capabilities of UAV-based hyperspectral imaging to enable water quality monitoring in real-time scenarios where contaminant sources may not be known in advance requires two additional capabilities: dimensionality reduction techniques to permit the visual comparison of HSI, and endmember extraction techniques which can identify spectral signatures corresponding to unique sources within the imaging scene. In remote sensing where reference data are typically sparse, many such techniques have been developed. Principal component analysis and t-distributed stochastic neighbor embedding are popular dimensionality reduction approaches commonly used to reduce HSI to two or three dimensions for visualization \cite{tyo2003principal,zhang2015hyperspectral}. For endmember extraction there are a variety of approaches including geometric methods like vertex component analysis, statistical methods like k-nearest neighbors and non-negative matrix factorization, and deep learning methods based on autoencoder architectures \cite{heylen2014review,nascimento2005vertex, Feng2022HyperspectralUB, cariou2015unsupervised, su2019daen, borsoi2019deep, palsson2020convolutional}. Ultimately, an ideal approach should enable both visualization and identification of relevant spectral endmembers.

Among the numerous unsupervised approaches, Kohonen's self-organizing map (SOM) is an attractive option which 
simultaneously accomplishes both tasks. 

provides a low dimensional representation of data into 

SOM have been used 

However, the SOM suffers from key limitations

\cite{cantero2004analysis} -- use SOM for endmember extraction
\cite{som-satellite} -- SOM For synoptic scale wind and sea surface temperature patterns
\cite{som-hsi} -- SOM for HSI feature extraction
\cite{danielsen2021self} -- SOM for compression of HSI from cubesat

To address these shortcomings, Bishop, Svensen, and () developted the Generative Topographic mapping (GTM) \cite{gtm-bishop-1}. The GTM is a fully Bayesian realization of the SOM which 



In this paper, we 

% 4. SOM 

% - SOM is an attractive approach as it enables both dimensionality reduction and unsupervised classification. 
% - Furthermore, the SOM maintains a topological relationship between learned classes such that classes near each other in the SOM grid are more similar than classes far apart
% - SOM can be used to nonlinearly map data into a 2-dimensional space
% - Weight vectors corresponding to each SOM node can be interpreted as spectral endmembers. The training algorithm updates weights based on euclidean distance between data record and each node weight vector
% - Add citations and examples of use in remote sensing / HSI
% - 

% Among unsupervised methods, Kohonen's self-organizing map provides an attractive compromise enabling both dimensionality reduction and classification...

% 5. GTM

% 6. This paper...


% Using data collected at site in Montague, North Texas, we demonstrate the ability of this paradigm with two case studies. First, we demonstrate how the GTM can be used to provide a high-resolution unsupervised classification of a water body enabling the rapid identification of relevant regions which can then be used to direct further investigation by reference-sensor equipped boats. Second, we demonstrate the ability to perform endmember extraction and abundance mapping using imagery from a test-release of rhodamine dye. Together with the normalized spectral similarity score, these spectral signatures are identified and their associated area in the scene is determined.




\bibliographystyle{plain}
\bibliography{paper/references.bib}

\end{document}