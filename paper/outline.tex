\documentclass{article}
\usepackage{graphicx} % Required for inserting images
\usepackage[margin=0.75in]{geometry}
\usepackage{outlines}


\begin{document}

\section*{Introduction}
\begin{outline}[enumerate]
\1 Challenges for water quality characterization
    \2 Traditional remote sensing suffers from limited spatial, spectral, and temporal resolution
        \3 Papers: 
            \4 Broad bands of traditional remote sensing multi-spectral imagers struggle to discern water quality parameters like chlorophyll-a in highly turbid waters due to domintation of spectral signal by sediments. \cite{ritchie2003remote}
            \4 Small, irregular shape of Finnish lakes makes analysis using medium resolution satellite data challenging due to reflectance from shore and near-shore vegetation. "If even a small protion of a pixel is covered by land the retrieval of water quality variables may not be possible" \cite{koponen2002lake}
    \2 This can be addressed by combining two technologies: 
        \3 Hyperspectral Imaging (improved spectral resolution)
        \3 Unmanned Aerial Vehicles (improves spatial and temporal resolution)
        \3 Studies using UAV + HSI for water quality: "UAV-based HSI has emerged as a promising tool..."
            \4 UAV-based HSI addresses resolution, signal-to-noise, and cloud cover issues that plague traditional satellite and high-altitude platforms\cite{banerjee2020uav}
            \4 UAV-based HSI for ecological monitoring \cite{arroyo2019implementation}
    \2 Remaining challenge: how can we extract meaningful insights from complicated spectral data?
        \3 "Ability for measuring hundreds of bands raises complexity when considering the sheer quantity of acquired data, whose usefulness depends on both calibration and corrective tasks occruing in pre- and post-flight stages" \cite{adao2017hyperspectral}
\1 Current approaches for characterization
    \2 Band ratios and ML for water quality parameter inversion
        \3 Papers:
            \4 UAV-based HSI with embedding processing capabiltieis enables rapid generation of spectral indices like the NDVI \cite{horstrand2019uav}
            \4 UAV-based multispectral imaging used to map turbidity (Secchi Disk Depth) \cite{vogt2016near}
            \4 Polynomial combinations of HSI bands from UAV-mounted HSI imagery used to map turbidity, total nitrogen, total phosphorus, ammonia nitrogen, permanganate index, and chemical oxygen demand \cite{zhang2022selection}
            \4 ML methods used to map UAV-based HSI to chlorophyll-a and suspended solids. Tree-based methods worked best (Catboost) \cite{lu2021retrieval}
            \4 Estimation of CDOM, Chlorophyll-a, turbidity, diatoms, and green-algae from HSI using Machine Learning with hyperspectral sensor and two in-situ sensor probes \cite{keller2018hyperspectral}
            \4 ICA and SVM used for HSI classification (not water) \cite{liang2010hyperspectral}
            \4 SVA and LDA used for HSI classification (not water) \cite{zhangSVM2014}
    \2 Key limitation: need for in situ reference data
        \3 For remote sensing applications, collection of datasets relies on serendipitous satellite overpasses. Acquiring sufficient volumes of data needed to calibrate and evaluate inversion models can require decades of observations \cite{aurin2018remote,ross2019aquasat}
        \3 Can be addressed by extending the UAV approach to coordinated robot teams including autonomous in situ sensing \cite{robot-team-1, robot-team-2}
    \2 Remaining limitation: insights are limited by available reference sensors 
        \3 Sensor require careful calibration, for example using specific standards which may not be relevant for local water
    \2 Need: unsupervised methods to extract meaningful information from full HSI data
        \3 Useful for deciding where/how to collect reference data
        \3 Useful for time-critical applications where identification of anomolous sources is important e.g. oil spills, industrial waste, etc... 
\1 Unsupervised ML Approaches
    \2 Dimensionality reduction
        \3 PCA is a popular choice but is a linear model
            \4 Mineral endmembers identified from HSI using sparse PCA \cite{yousefi2016mineral}
        \3 t-SNE can be computationally complex 
    \2 Unsupervised Classification
        \3 k-NN, fuzzy c-means, DBSCAN, etc... matrix factorizations
        \3 Can be helpful for identifying clusters, but don't aid in visualization by themselves
            \4 Robust Manifold Matrix Factorization used for dimensionality reduction \textit{and} clustering \cite{zhang2019hyperspectral}
    \2 Endmember extraction and Spectral Unmixing
        \3 Geometric and statistical methods
        \3 Can be useful for identifying relevant endmembers but may rely on presence of pure pixels
            \4 This assumption may not be valid for water leafing reflectance spectra due to interactions of suspended solids, dissolved components, etc...
        \3 Once endmembers are identified, abundance maps can be produced using spectral angle mapper (e.g. cosine distance) or unmixing models
            \4 However, this does not itself to visualization
    \2 Need: a (principled) method which simultaneously enables dimensionality reduction \textit{and} classification/endmember extraction.
\1 Self Organizing Map and Generative Topographic Mapping
    \2 Self Organizing map developed by Kohonen based on principle that \textit{neurons that fire together, wire together}.
        \3 Construct an unsupervised classification into $N=n\times n$ classes
        \3 Classes are arranged in a regular grid of typically two dimensions with different possible topologies:
            \4 square vs hexagonal lattice (number of neighbors)
            \4 rectangular, cylindrical, toroidal topologies based on boundary conditions
        \3 The map is trained to enforce similarity between neighboring nodes, that is, classes near each other in the SOM grid are more similar than nodes that are far apart. 
        \3 In the context of HSI and water quality, the SOM grid can be used to perform dimensionality reduction via the location of each datapoint in the SOM grid 
        \3 SOM nodes can \textit{also} be interpreted as endmembers/classes
            \4 Add citations here, e.g. David's dust sources paper and other SOM related HSI studies
            \4 SOM used for endmember extraction with HSI for water quality \cite{cantero2004analysis}
    \2 Drawbacks of SOM
        \3 Training algorithm is based on heuristic approach
        \3 Does not provide a clear probabilistic interpretation or a principled approach for hyperparameter choice
    \2 The Generative Topographic Mapping addresses these issues
        \3 Fully Bayesian re-interpretation of the SOM
        \3 Enables both dimensionality reduction \textit{and} classification like the SOM
        \3 Uses of the GTM in HSI:
\1 Paper overview
    \2 Development of the GTM as an efficient approach for dimensionality reduction and unsupervised classification of UAV-acquired HSI
    \2 Apply the GTM to data collected at a pond in Montague, North Texas
    \2 Train a GTM on water-only pixels to provide detailed segmentation water spectra
    \2 Use GTM on combined dataset for demonstration of endmember extraction and source mapping
        \3 Algal abundance (i.e. application to algal bloom)
            \4 Comment on potential application for algal species discrimination
        \3 Rhodamine Dye Tracer simulating dispersion of pollution source
            \4 Application to oil spill identification
        
\end{outline}
    
\section*{Materials and Methods}
\section*{Results}
\section*{Discussion}
\section*{Conclusions}

\bibliographystyle{plain}
\bibliography{paper/references.bib}

\end{document}