\documentclass{article}
\usepackage{graphicx} % Required for inserting images
\usepackage[margin=0.75in]{geometry}
\usepackage{outlines}


\begin{document}

NOTE: We should move the description of the GTM algorithm to the top of the methods section.

\section*{Outline}
We should recast the introduction to focus on unsupervised classification rather than endmember extraction... 

Characterization on inland waters via remote sensing is hard because (reasons). Hyperspectral imaging has emerged as a pivotal technology in this space due to the dramatic improvement in spectral resolution afforded by HSI platforms. Specifically, the combination with hyperspectral imaging together with autonomous collection by UAVs enables unparalleled spectral, spatial, and temporal resolution. Significant effort has developed the use of hyperspectral imagery for the inversion of optically-active water quality parameters. However, this approach requires access to sufficient quantities of in situ reference data to fit and evaluate each model. Traditionally, this process has required collocation of remote sensing imagery with data from field campaigns spanning decades (cite Aurin and Ross). Recently, we have demonstrated how combining autonomous collection of in situ data with UAV-borne hyperspectral imagery dramatically reduces the time required to collect these data. Both of these approaches require we already know what to look for. 




\begin{outline}[enumerate]
\1 Introduction
    \2 Characterization of inland water quality via remote sensing is challenging for 4 key reasons:
        \3 Key limitations:
            \4 Spectral resolution (visible and multi-spectral imaging)
            \4 Spatial Resolution (large pixel sizes)
            \4 Temporal Resolution (long satellite revisit times)
            \4 Availability of in situ reference data
        \3 When appropriate data are available, supervised ML approaches can map reflectance from remote sensing imagery to relevant water quality parameters
    \2 Hyperspectral Imaging addresses spectral resolution
    \2 UAV addresses spatial and temporal resolution
    \2 Robot team addresses in situ data... but what do we do if we don't already know which sensors to use? 
    \2 Unsupervised ML, Endmember extraction, and Spectral Unmixing
        \3 k-NN, fuzzy c-means, matrix factorizations, etc.
            \4 Are general but do not provide relationship between classes e.g. is class 1 more similar to class 2 or class 10?
        \3 SOM addresses this by enforcing a topological relationship between clusters. Classes close to each other in the map are more similar than classes far apart
        \3 Similarly, we may also be interested in the spectral characteristics of the learned *classes*, i.e. we want to think of the classes as *endmembers*.
            \4 Traditional approaches rely on assumption of existence of "pure pixels" in the scene
            \4 additionally, we may have to treat number of endmembers as a hyperparameter

    \2. ...in this paper, we explore how the generative topographic mapping can be used to
        \3 enable principled unsupervised classification of hyperspectral imagery by
            \4 Bayesian treatment (as compared to the self organizing map)
            \4 topological ordering with smooth relationship between classes in latent space
        \3 Train GTM on water pixels to obtain unsupervised classification mapping the detailed structure and small-scale variability of pond in Montague, North Texas
        \3 Mapping can be further utilized for smart deployment of robot team (maybe save for discussion section?) i.e. requisition boat to bring reference sensors to "interesting" area
        \3 Using combined dataset including land, water, and tracer dye, we demonstrate endmember extraction
        \3 Using learned spectra together with NS3, we are able to evaluate extent of algae (application to harmful algal bloom mapping)
        \3 Use learned spectra to map extent


\1 Materials and Methods
    \2 Overview paragraph about project, i.e. plan is to use the Generative Topographic Mapping to analyze drone-based hyperspectral imagery collected at a pond in Montague North Texas on 23-November 2021. 
   \2 Autonomous Robot Team
        \3 Describe drone setup
            \4 Alta X Quadcopter
            \4 Resonon Pika SC2 HSI
                \begin{itemize}
                    \item 462 wavelengths per pixel
                    \item 1600 pixels per scan-line
                    \item embedded GPS/IMU
               \end{itemize}
             \4 Intel NUC for on-board compute
   \2 Data Collection
        \3 Describe collection site in Montague, North Texas
        \3 Collection was performed near solar noon to maximize incident light. At this time of year, the maximum solar elevation was $\approx 60^\circ$ so that sun-glint is not as much of an issue. (Cite sun-glint paper here).
    \2 Pre-processing
        \3 Reflectance conversion of radiance imagery using on-board downwelling irradiance spectrometer
        \3 Georectification of collected
        \3 Wavelength selection to $\lambda \leq 900 \text{nm}$ as longer wavelengths are noisy
        \3 Rescaling of spectra so that peak has $\tilde{R}  = 1.0$  to account for variability in incident lighting conditions due 
    \2 Generative Topographic Mapping
        \3 Start off with Kohonen SOM paper. Key benefit:
            \4 Topological relationship between classes i.e. similarity vs dissimilarity as a function of distance in mapping space (latent space)
        \3 Limitations of SOM
            \4 Not probabilistic
            \4 No clear convergence criterion
            \4 Heuristic based approach with no clear way to select hyperparameters like neighborhood function, number of classes, etc...
            \4 Heskes did reinterpret algorithm as stochastic gradient descent on a cost function...
        \3 GTM Algorithm
            \4 Describe Algorithm
            \4 Describe parameters
        \3 Bayesian information criterion and Akaike information criterion for hyperaprameter optimization
            \4 As an alternative to performing full Bayesian treatment for marginalization over hyperparameters
        \3 Abundance mapping with spectral similarity score
            \4 paper comparing different distance functions for spectra
                \begin{itemize}
                    \item curse of dimensionality can be pernicious here as we have > 462 wavelengths per pixel
                \end{itemize}
            \4 NS3 offers good trade off between spectral angle and euclidean-like distance (MSE)
        \3 Describe use of thresholded NS3 for mapping abundance of GTM derived endmembers.
    \2 Overview of two approaches
        \3 GTM trained on water-only pixels for unsupervised classification of pond
            \4 Goal: Reveal the small-scale spatial variability of pond water
            \4 Discuss how unsupervised classification can help inform autnomous data collection by robot team (save for discussion?)
        \3 GTM trained on combined dataset including water, land, and dye plume pixels
            \4 Fit GTM
            \4 Identify spectral signatures of key exemplar spectra (endmembers)
            \4 Use endmembers to map algal abundance and time evolution of rhodamine dye plume
\1 Results
    \2  GTM fit on Full Dataset
    \2 Hyperparameter optimization
    \2 Spectral signature identification
    \2 Water Class Map (water-only GTM)
    \2 Algae Identification
    \2 Dye plume identification

\1 Discussion
    \2 Overview of GTM applications in our paper
        \3 Unsupervised classifciation of entire pond (no-dye)
        \3 Endmember extraction of spectral signatures
    \2 Comparison to existing approaches
        \3 unsupervised classification methods (to general, no relationship between classes) 
        \3 endmember extraction (mostly linear, assume existence of pure spectra)
        \3 deep learning approaches to endmember extraction don't provide clear way to asses number of endmembers needed
    \2 Applications of either of these with drones?
        \3 Are there any papers using unsupervised methods or endmember extraction with drone based imagery (HSI or otherwise)?
    \2. Use of GTM as component of Robot Team
        \3 As a feature transformer / preprocessing step for supervised methods (as opposed to PCA, k-NN, etc...)
        \3 identification of "interesting" regions for smart provisioning of robot boat
            \4 Standard approach is to survey in a grid
            \4 Can use classification map with real-time measurement to ensure optimal collection of data by collecting samples with from as many GTM classes as possible
    \2 Extensions and future work
        \3 Incorporation of GTM extensions for efficient online-learning
            \4 Comment on time to train standard GTM
                \begin{itemize}
                    \item Key parameters are size of dataset (number of records) and number of RBF centers, $M$.
                \end{itemize}
            \4 Bishop suggest extension for on-line training of GTM using batches.
            \4 All of this could be done in the field limiting the time and cost of surveying as well as optimizing the time-to-insights
        \3 Utilize GTM with further in situ data collection to identify spectral signatures of specific algal species for HAB detection
        \3 Same but for crude oil types...

\1 Conclusions




\end{outline}

\bibliographystyle{plain}
\bibliography{paper/references.bib}

\end{document}
